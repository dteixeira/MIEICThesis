%-----------------------------------------------
% Template para criação de resumos de projectos/dissertação
% jlopes AT fe.up.pt,   Fri Jul  3 11:08:59 2009
%-----------------------------------------------

\documentclass[9pt,a4paper]{extarticle}

%% English version: comment first, uncomment second
\usepackage[portuguese]{babel}  % Portuguese
%\usepackage[english]{babel}     % English
\usepackage{graphicx}           % images .png or .pdf w/ pdflatex OR .eps w/ latex
\usepackage{times}              % use Times type-1 fonts
\usepackage[utf8]{inputenc}     % 8 bits using UTF-8
\usepackage{url}                % URLs
\usepackage{multicol}           % twocolumn, etc
\usepackage{float}              % improve figures & tables floating
\usepackage[tableposition=top]{caption} % captions
%% English version: comment first (maybe)
\usepackage{indentfirst}        % portuguese standard for paragraphs
%\usepackage{parskip}

%% page layout
\usepackage[a4paper,margin=30mm,noheadfoot]{geometry}

%% space between columns
\columnsep 12mm

%% headers & footers
\pagestyle{empty}

%% figure & table caption
\captionsetup{figurename=Fig.,tablename=Tab.,labelsep=endash,font=bf,skip=.5\baselineskip}

%% heading
\makeatletter
\renewcommand*{\@seccntformat}[1]{%
  \csname the#1\endcsname.\quad
}
\makeatother

%% avoid widows and orphans
\clubpenalty=300
\widowpenalty=300

\begin{document}

\title{\vspace*{-8mm}\textbf{\textsc{Uma Plataforma Computacional\\para Análise de Expressão Génica}}}
\author{\emph{Diogo André Rocha Teixeira}\\[2mm]
  \small{Dissertação realizada sob a orientação do \emph{Prof.\ Rui Camacho} e co-orientação de \emph{Nuno Fonseca}}\\
\small{na \emph{Faculdade de Engenharia da Universidade do Porto}}}
\date{}
\maketitle
%no page number 
\thispagestyle{empty}

\vspace*{-4mm}\noindent\rule{\textwidth}{0.4pt}\vspace*{4mm}

\begin{multicols}{2}

\section{Contexto} \label{sec:context}

A biologia molecular é um ramo da biologia que estuda as atividades biológicas
dos seres vivos, ao nível molecular. As bases para esta área de estudo foram
criadas no início da década de 1930, embora apenas tenha emergido na sua forma
mais moderna na década de 1960, com a descoberta da estrutura do DNA. Entre os
processos estudados por este ramo da biologia está a expressão génica. A
expressão génica é o processo através do qual moléculas de DNA são
transformadas em produtos genéticos úteis, tipicamente proteínas, que são
essenciais para os organismos vivos. Este conhecimento não é apenas importante
em áreas como biologia molecular ou evolutiva, mas tem aplicações cruciais em
áreas como medicina. Um exemplo de uma destas aplicações é a utilização de
análise de expressão génica no diagnóstico e tratamento de pacientes com cancro
\cite{Pusztai01062003}.

Com o advento das técnicas de \textit{Next Generation Sequencing} (NGS) os
investigadores têm à sua disposição grandes quantidades de dados de
sequenciação, cuja produção é mais barata e rápida. Estes dados podem ser usados
para obter informação relevante sobre a expressão génica de organismos. Mas, à
medida que o custo da sequenciação de genomas é reduzido, o custo do
processamento dessa informação aumenta. Técnicas NGS costumam produzir
\emph{reads}\footnote{Uma \emph{read} é um fragmento de um genoma/transcriptoma,
obtido através de técnicas de sequenciação.} mais curtas quando comparadas com
aquelas produzidas por técnicas anteriores, apresentando um problema mais
desafiante, de ponto de vista computacional \cite{Wolf2013}.

\section{Problem de Domínio} \label{sec:problem}

Apesar dos grandes avanços nas últimas décadas, a biologia molecular é ainda uma
área recente e, como tal, ainda existem muitas incógnitas e conhecimento
parcial. Alguns dos mecanismos reguladores da expressão génica são ainda
desconhecidos. Um destes mecanismos regula a velocidade de transcrição de RNA.
Um dos objetivos desta dissertação é preceber de que forma é que as sequências
finais do exões de um gene afetam a velocidade com que estes são transcritos. O
outro objetivo é perceber de que forma a manipulação das \emph{RNA-binding
proteins} (RBP) pode ser usada para melhor perceber a expressão génica de um
organismo. Estas são tarefas complexas que podem ser decompostas nos três
principais problemas que vão ser endereçados nesta dissertação.

  \textbf{Alinhamento de \emph{reads} contra um genoma de referência e análise
  da expressão diferencial entre amostras de diferentes indivíduos}. Este é o
  problema mais complexo tratado nesta dissertação. Serão usados dados obtidos
  através de uma técnica de sequenciação donominada \emph{RNA
  Sequencing}.

  \textbf{Enriquecimento de genes e análise de RBPs}. Esta parte do trabalho
  tem como objetivo recolher informação relevante sobre os genes em estudo, de
  forma a ajudar os biologistas a melhor perceber a função desses genes.
  Informação sobre RBPs é particularmente importante para manipular genes e
  muito útil para melhor perceber a expressão génica.

  \textbf{Análise dos dados produzidos, usando técnicas de \emph{data mining},
  mais especificamente técnicas de \emph{clustering}}. Estas técnicas serão
  aplicadas num esforço de dar aos biologistas mais informação relevante sobre
  expressão génica, descobrindo possíveis relações implícitas nessa informação.

Resolver estes problemas requer o uso de ferramentas computacionais. Como tal,
o desenvolvimento de um sistema informático (ou vários sistemas) para resolver
estes problemas surge como um objetivo secundário da dissertação.

\section{Motivação e Objetivos} \label{sec:motivation}

A análise de expressão génica é essencial para a biologia molecular moderna.
Entre muitas das possíveis aplicações desta informação, podemos destacar:
melhor classificação e diagnóstico de doenças; avaliação a reação de células a
um tratamento específico.

Embora existam hoje ferramentas computacionais poderosas para resolver inúmeros
problemas de biologia, muitas dessas ferramentas exigem um conjunto muito
específico de competências técnicas e tem uma curva de aprendizagem íngreme.
Assim, a motivação mais importante por trás desta dissertação é criar
ferramentas mais fáceis de utilizar. Isto significa desenvolver um sistema
simples, para que qualquer utilizador possa aprender a operá-lo num curto espaço
de tempo, com pouco esforço; O sistema deve também ser suficientemente avançado
para atender às necessidades do utilizador.

Outro problema é a dispersão da informação e a tarefa repetitiva e prolongada de
compilar essa informação. Frequentemente é necessário juntar manualmente
informação proveniente de um grande número de plataformas, que usam formatos e
notações inconsistentes. O segundo objectivo é, portanto, desenvolver um sistema
que seja capaz de realizar esta tarefa pelo utilizador, tornando o processo mais
rápido e simples.

\section{Projeto} \label{sec:project}

O projecto revolve em torno do desenvolvimento do protótipo de um sistema
informático capaz de resolver os problemas acima mencionados. Devido à
complexidade deste sistema, o seu desenvolvimento seguiu uma organização
modular. O sistema foi assim dividido em três modulos.

  \textbf{O \emph{pipeline} de análise de expressão diferencial} é responsável
  por alinhar \emph{reads} contra um genoma de referência, comparando depois os
  resultados para diferentes amostras. O \emph{pipeline} é baseado na ferramenta
  iRAP. As funcionalidades do \emph{pipeline} são ainda reforçadas com
  ferramentas para configuração automática de experiências e consolidação de
  resultados de expressão diferencial.

  \textbf{O fluxo de trabalho de análise de RBPs} agrega informações sobre RBPs
  de várias plataformas biológicas (Ensembl, NCBI, UniProt, etc.), organizando
  essa informação de maneiras que são úteis para os investigadores. Além disso,
  esta informação é agrupada usando técnicas de \emph{data mining}, a fim de
  revelar os grupos de genes e RBPs que podem ter relevância biológica.

  \textbf{A plataforma web} é responsável por armazenar e gerir dados genéticos,
  coordenar a interação entre os outros componentes do sistema e fornecer uma
  interface web para interação com os utilizadores.

\section{Caso de Estudo}

Dois casos de estudo foram realizados a fim de avaliar a qualidade do sistema
desenvolvido.

O primeiro caso de estudo focou as melhorias que foram desenvolvidos para o
\emph{pipeline} de análise de expressão génica. Foi baseado numa experiência
passada (experiência ArrayExpress \emph{E-GEOD-48829}) que estudou a bactéria
\emph{E. coli}. O objetivo deste caso de estudo foi verificar se a ferramenta
desenvolvida poderia ajudar a melhorar a confiança dos investigadores nos
resultados de expressão diferencial, combinando os melhores resultados de várias
ferramentas. Os resultados foram comparados com os resultados brutos e os
resultados filtrados por \emph{p-value}\footnote{O \emph{p-value} é usado para
aferir a significância estatística de resultados \cite{goodman45dirty}.}.
Conclui-se que a ferramenta reduziu significativamente o número de genes nos
resultados, aumentando a confiança nesses resultados, dando ao mesmo tempo um
conjunto de genes menos extenso e portanto mais fácil de analisar em
experiências futuras.

O segundo caso de estudo foi realizado em colaboração com especialistas do
Instituto de Biologia Molecular e Celular (IBMC). O conjunto de dados estudado
era composto por vinte e três genes da família \emph{RhoGTPase}, do genoma de
\emph{Rattus norvegicus} (rato norueguês). Este caso de estudo teve três
objetivos principais: avaliar a utilidade geral da ferramenta de análise de
RBPs; comparar a solução desenvolvida com as já existentes; e avaliar o impacto
do desempenho dos computadores no desempenho global da ferramenta. Os resultados
obtidos foram verificados em termos da sua integridade,  correção e relevância
do ponto de vista da biologia. A validação biológica foi realizada por
especialistas IBMC. Concluiu-se que com a ferramenta desenvolvida é possível
obter os mesmos resultados que um especialista obteria, numa fracção do tempo e
disponibilizando mais informação útil.

\section{Conclusões}

Os nossos objetivos, do ponto de vista do estudo do problema e esboço de uma
solução, foram totalmente cumpridos. A solução proposta corresponde a todas as
nossas expectativas. No entanto a implementação do sistema de análise e
alinhamento de dados RNA-Seq não foi totalmente concluída, devido a limitações
de tempo. Como tal, o nosso objectivo de criar e testar um protótipo do sistema
completo não foi atingido.

\section{Trabalho Futuro}

A continuação do trabalho proposto passaria por terminar a implementação e
integração do \emph{pipeline} de análise de dados de RNA-Seq. Isso permitiria à
nossa solução funcionar conforme foi projetada, integrando todo o processo de
análise. Além disso, seria interessante estudar as ferramentas desenvolvidas em
termos de desempenho, quando usadas com grandes volumes de informação e pedidos.
Embora as ferramentas tenham sido desenvolvidas tendo em consideração o seu
desempenho, o seu funcionamento em larga escala necessita de outro tipo de
infra-estrutura, que não foi considerada nesta dissertação.

%%English version: comment first, uncomment second
\bibliographystyle{unsrt-pt}  % numeric, unsorted refs
%\bibliographystyle{unsrt}  % numeric, unsorted refs
\bibliography{refs}

\end{multicols}

\end{document}
