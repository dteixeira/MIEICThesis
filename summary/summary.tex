%-----------------------------------------------
% Template para criação de resumos de projectos/dissertação
% jlopes AT fe.up.pt,   Fri Jul  3 11:08:59 2009
%-----------------------------------------------

\documentclass[9pt,a4paper]{extarticle}

%% English version: comment first, uncomment second
%\usepackage[portuguese]{babel}  % Portuguese
\usepackage[english]{babel}     % English
\usepackage{graphicx}           % images .png or .pdf w/ pdflatex OR .eps w/ latex
\usepackage{times}              % use Times type-1 fonts
\usepackage[utf8]{inputenc}     % 8 bits using UTF-8
\usepackage{url}                % URLs
\usepackage{multicol}           % twocolumn, etc
\usepackage{float}              % improve figures & tables floating
\usepackage[tableposition=top]{caption} % captions
%% English version: comment first (maybe)
\usepackage{indentfirst}        % portuguese standard for paragraphs
%\usepackage{parskip}

%% page layout
\usepackage[a4paper,margin=30mm,noheadfoot]{geometry}

%% space between columns
\columnsep 12mm

%% headers & footers
\pagestyle{empty}

%% figure & table caption
\captionsetup{figurename=Fig.,tablename=Tab.,labelsep=endash,font=bf,skip=.5\baselineskip}

%% heading
\makeatletter
\renewcommand*{\@seccntformat}[1]{%
  \csname the#1\endcsname.\quad
}
\makeatother

%% avoid widows and orphans
\clubpenalty=300
\widowpenalty=300

%% Path to the figures directory
\graphicspath{{figures/}}

\begin{document}

\title{\vspace*{-8mm}\textbf{\textsc{A Computational Platform\\for Gene Expression Analysis}}}
\author{\emph{Diogo André Rocha Teixeira}\\[2mm]
  \small{Dissertation conducted under the supervision of \emph{Prof. Rui Camacho} and co-supervision of \emph{Nuno Fonseca}}\\
\small{at \emph{Faculdade de Engenharia da Universidade do Porto}}}
\date{}
\maketitle
%no page number 
\thispagestyle{empty}

\vspace*{-4mm}\noindent\rule{\textwidth}{0.4pt}\vspace*{4mm}

\begin{multicols}{2}

\section{Context} \label{sec:context}

Molecular biology is a branch of biology that studies biological activities of
living beings, at a molecular level. The grounds for this field of study were
set in the early 1930s, although it only emerged in its modern form in the
1960s, with the discovery of the structure of DNA. Among the processes studied
by this branch of biology is gene expression. Gene expression is the process by
which DNA molecules are transformed into useful genetic products, typically
proteins, which are essential for living organisms. This knowledge is not only
important in fields like evolutionary or molecular biology, but has crucial
applications in fields such as medicine. One example of such an application is
the usage of gene expression analysis in the diagnosis and treatment of cancer
patients \cite{Pusztai01062003}.

With the advent of \textit{Next Generation Sequencing} (NGS) techniques
researchers have at their disposal huge amounts of sequencing data, that is not
only cheaper and faster to produce, but also more commonly available. This data
can then be used to obtain relevant information about organisms' gene
expression. But, as the cost of sequencing genomes was reduced, the cost of
processing such information was increased. NGS techniques tend to produce much
smaller reads\footnote{A \textit{read} is a single fragment of a
genome/transcriptome, obtained through sequencing techniques.} than previously
used techniques, presenting a more challenging problem, from a computational
standpoint \cite{Wolf2013}.

\section{Domain Problem} \label{sec:problem}

Despite its great advancements in the past decades, molecular biology is still a
relatively new subject and, as such, there are still some unknowns and partial
knowledge in this area. In respect to gene expression, some mechanisms of this
intricate process are yet to be fully understood. One such mechanism is the one
that regulates the transcription speed into RNA. One of the objectives of this
thesis is to understand how the final sequences of a gene's exons are
responsible for the speed at which the exons themselves are transcribed. The
other objective is to understand how RNA-binding protein (RBP) manipulation can
be used to better understand an organism's gene expression. These are, however,
complex tasks that can be further decomposed in the three main problems that
will be addressed in the thesis, namely:

  \textbf{Sequencing read alignment against a reference genome and differential
  expression analysis between samples of different individuals} (of the same
  species). This is effectively one of the most complex problems addressed in
  the thesis. We will use data obtained through a sequencing method called RNA
  Sequencing\footnote{RNA Sequenceing (RNA-Seq) is also referred to as
  \textit{Whole Transcriptome Shotgun Sequencing}, or WTSS.}.

  \textbf{Gene enrichment and RBP analysis}. This part of the work aims to
  collect as much relevant information as possible about the particular genes
  being studied at the time, to help biologists to better understand their
  function. RBP knowledge is particularly important for gene manipulation and a
  very useful tool for better understanding gene expression.

  \textbf{Further analysis of the produced data, using machine learning
  techniques for data mining, specifically for clustering analysis}. These
  techniques will be employed in an effort to give biologists more relevant
  information about gene expression, uncovering possible relationships in the
  retrieved information.

Solving these problems requires the use of computational tools. As such, the
development of a computer system (or multiple systems) to tackle these problems
emerges as a secondary objective of the thesis.

\section{Motivation and Objectives} \label{sec:motivation}

Gene expression analysis is essential for modern day molecular biology. Among
many of the possible applications of this information, we can highlight: better
classification and diagnosis of diseases, assessing how cells react to a
specific treatment, and others.

While nowadays powerful computational tools exist to target almost any biology
problem, many of those tools require a very specific set of technical skills and
have a steep learning curve. Possibly the most important motivation behind this
thesis, and ultimately its main objective, is to provide researchers with
powerful yet simple and user friendly tools. This means developing a system
simple enough that any user can learn to operate it in a short period of time
with minimal effort, but sufficiently advanced to suit the user's research
needs.

Another typical problem that biology researchers face nowadays is information
dispersion and the repetitive and lengthy task of compiling that information.
Researchers frequently have to manually join information originating from a
multitude of different platforms, which use inconsistent formats and notations.
Our second objective is therefore to provide a system that is able to take this
burden off the user, making the process faster and simpler.

\section{Project} \label{sec:project}

The project itself revolves around the development of a prototype computer
system, capable of solving the aforementioned problems. Due to the complexity
of the complete system, its development followed a modular organization. The
envisioned system architecture is divided into three major components.

  \textbf{The differential expression analysis pipeline}
  is responsible for aligning reads against a reference genome and compare
  contrasts between different samples. The pipeline is based on the preexisting
  iRAP pipeline. The pipeline's capabilities are further enhanced with both job
  configuration automation and differential expression results consolidation
  (combining results from multiple differential expression tools).

  \textbf{The RNA-binding protein analysis workflow}
  aggregates information about RBPs from multiple biologic web databases
  (Ensembl, NCBI, UniProt, etc.) and organizes it in ways that are useful to
  biology researchers. Moreover, this information is clustered using data mining
  techniques, in order to reveal groups of genes and RBPs that may hold biologic
  relevance.

  \textbf{The web platform}
  is responsible for storing and managing genetic data, coordinating interaction
  between the other components of the system and providing a web interface for
  user interaction. This component is based mainly on typical web technologies,
  that is, a document based database for data storage (MongoDB), a web framework
  for business logic implementation (Padrino) and web markup and styling
  languages for interface implementation (HTML, CSS).

\section{Case Study}

A case study was conducted, in collaboration with IBMC (\emph{Instituto de
Biologia Molecular e Celular}) experts. The studied data set was composed by
twenty three genes from \emph{RhoGTPase} family, from \emph{Rattus norvegicus}
(commonly known \emph{norway rat}).

The obtained results were validated both in terms of their ... and their
biological correction and relevance. The biological validation was also
performed by IBMC experts. We concluded that the developed tool could mimic the
same results an expert would obtain, in a fraction of the time (see Tab.
\ref{tab:stress}) and providing much more useful information.

\begin{table}[H]
  \centering
  \caption{Result time comparison between manual analysis (done by an expert) and both test machines.}
  \begin{tabular}{l|lll}
    & \textbf{machine1} & \textbf{machine2} & \textbf{Expert} \\ \hline
    \textbf{100 IDs}   & $9m\ 56s$          & $11m\ 1s$      & $\approx 50h$\\
    \textbf{500 IDs}   & $41m\ 47s$         & $55m\ 51s$     & $\approx 250h$\\
    \textbf{900 IDs}   & $1h\ 33m\ 32s$     & $2h\ 7m\ 4s$   & $\approx 450h$\\
  \end{tabular}
  \label{tab:stress}
\end{table}

\section{Conclusions}

Our objectives, in terms of studying the problem at hand and developing a
solution to it, were completely fulfilled. The proposed solution corresponds to
all of our expectations. However, as previously discussed, the implementation of
the RNA-Seq data analysis pipeline system was not completed, due to time
constraints. As such, our objective of prototyping and testing the complete
system could not be completely achieved.

\section{Future Work}

The obvious continuation of the proposed work would be to finish the
implementation and integration of the RNA-Seq data analysis pipeline. This would
allow our solution to work as designed, integrating the complete analysis
pipeline, from sequencing data to gene clustering and result visualization.
Furthermore, it would be interesting to study the developed tools in terms of
performance, under large volumes of information and requests. Whilst the tools
were developed taking in consideration their performance, making them available
in a large scale would take another kind of infrastructure.

%%English version: comment first, uncomment second
%\bibliographystyle{unsrt-pt}  % numeric, unsorted refs
\bibliographystyle{unsrt}  % numeric, unsorted refs
\bibliography{refs}

\end{multicols}

\end{document}
