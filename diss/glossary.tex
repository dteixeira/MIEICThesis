%% Glossary
\newglossaryentry{alternative_splicing}{name=Alternative splicing,description={
    Alternative splicing that causes a specific gene to be transcribed to
    several different RNA sequences, by means of different combinations of
    exons. This process is a major source of protein diversity in an organism
    \cite{doi:10.1146/annurev.biochem.72.121801.161720}
  },nonumberlist
}
\newglossaryentry{cdna}{name=cDNA,description={
    DNA which has been reverse transcribed using RNA as a template \cite{gloss}
  },nonumberlist
}
\newglossaryentry{cytoplasm}{name=Cytoplasm,description={
    The protoplasm of a cell contained within the cell membrane, but excluding
    the nucleus. It helps to move materials around the cell and is also
    responsible for dissolving cellular waste \cite{gloss}
  },nonumberlist
}
\newglossaryentry{exon}{name=Exon,description={
    The portions of a genomic DNA sequence which will be represented in the
    final, mature mRNA. Exons may include coding sequences, the \textit{5'}
    untranslated region or the \textit{3'} untranslated region \cite{gloss}
  },nonumberlist
}
\newglossaryentry{expression}{name=Expression,description={
    To \qt{express} a gene is to cause it to function. A gene which encodes a
    protein will, when expressed, be transcribed and translated to produce that
    protein. A gene which encodes an RNA rather than a protein (for example, a
    rRNA gene) will produce that RNA when expressed \cite{gloss}
  },nonumberlist
}
\newglossaryentry{gene}{name=Gene,description={
    A unit of DNA which performs one function. Usually, this is equated with the
    production of one RNA or one protein. A gene contains coding regions,
    introns, untranslated regions and control regions \cite{gloss}
  },nonumberlist
}
\newglossaryentry{genome}{name=Genome,description={
    The total DNA contained in each cell of an organism. There are somewhere in
    the order of a hundred thousand genes, including coding regions, \textit{5'}
    and \textit{3'} untranslated regions, introns, \textit{5'} and \textit{3'}
    flanking DNA \cite{gloss}
  },nonumberlist
}
\newglossaryentry{intron}{name=Intron,description={
    Introns are portions of genomic DNA which are
    transcribed (and thus present in the primary transcript) but which are later
    spliced out. Thus, they are not present in the mature mRNA \cite{gloss}
  },nonumberlist
}
\newglossaryentry{mrna}{name=mRNA,description={
    \qt{Messenger RNA} contains sequences coding for a protein. The term mRNA is
    used only for a mature transcript (with all introns removed), rather than
    the primary transcript in the nucleus \cite{gloss}
  },nonumberlist
}
\newglossaryentry{nucleus}{name=Nucleus,description={
    The nucleus is a membrane enclosed part of the cell \cite{gloss}. It
    contains the cell's genetic information, in the form of DNA and RNA
    molecules \cite{gloss}
  },nonumberlist
}
\newglossaryentry{rrna}{name=rRNA,description={
    \qt{Ribosomal RNA} describes any of several RNAs which become part of the
    ribosome, and thus are involved in translating mRNA and synthesizing
    proteins \cite{gloss}
  },nonumberlist
}
\newglossaryentry{shotgun_cloning}{name=Shotgun cloning,description={
    The process of randomly shearing an organism's genomic DNA and cloning it
    into a suitable vector, resulting in a genomic library \cite{gloss}
  },nonumberlist
}
\newglossaryentry{shotgun_sequencing}{name=Shotgun sequencing,description={
    Sequencing the DNA library created by shotgun cloning \cite{gloss}
  },nonumberlist
}
\newglossaryentry{signal_transduction}{name=Signal transduction,description={
    Process in which an extracellular signaling molecule activates a specific
    receptor located in the border (or inside) a cell; in turn this receptor
    triggers a chain of events inside the cell that leads to its response
    \cite{lodish2000molecular}
  },nonumberlist
}
\newglossaryentry{transcription}{name=Transcription,description={
    The process of copying DNA to produce an RNA transcript. This is the first
    step in the expression of any gene. The resulting RNA will produce the
    desired protein molecule by the process of translation \cite{gloss}
  },nonumberlist
}
\newglossaryentry{translation}{name=Translation,description={
    The process of decoding a strand of mRNA, thereby producing a protein based
    on the code \cite{gloss}
  },nonumberlist
}
\newglossaryentry{trna}{name=tRNA,description={
    \qt{Transfer RNA} represents one of a class of rather small RNAs used by the
    cell to carry amino acids to the enzyme complex (the ribosome) which builds
    proteins, using an mRNA as a guide \cite{gloss}
  },nonumberlist
}

%This brief glossary was based on a similar work by Robert Lyons \cite{gloss}.

%\begin{flushleft}
%\begin{tabular}{l p{0.8\linewidth}}

%cDNA                  & DNA which has been reverse transcribed using RNA as a
%template.\\

%Cytoplasm             & The protoplasm of a cell contained within the cell
%membrane, but excluding the nucleus. It helps to move materials around the cell
%and is also responsible for dissolving cellular waste.\\

%Exon                  & The portions of a genomic DNA sequence which will be
%represented in the final, mature mRNA. Exons may include coding sequences, the
%\textit{5'} untranslated region or the \textit{3'} untranslated region.\\

%Expression            & To \qt{express} a gene is to cause it to function. A gene
%which encodes a protein will, when expressed, be transcribed and translated to
%produce that protein. A gene which encodes an RNA rather than a protein (for
%example, a rRNA gene) will produce that RNA when expressed.\\

%Gene                  & A unit of DNA which performs one function. Usually, this
%is equated with the production of one RNA or one protein. A gene contains coding
%regions, introns, untranslated regions and control regions.\\

%Genome                & The total DNA contained in each cell of an organism.
%There are somewhere in the order of a hundred thousand genes, including coding
%regions, \textit{5'} and \textit{3'} untranslated regions, introns, \textit{5'}
%and \textit{3'} flanking DNA.\\

%Intron                & Introns are portions of genomic DNA which are
%transcribed (and thus present in the primary transcript) but which are later
%spliced out. Thus, they are not present in the mature mRNA.\\

%mRNA                  & \qt{Messenger RNA} contains sequences coding for a
%protein. The term mRNA is used only for a mature transcript (with all introns
%removed), rather than the primary transcript in the nucleus.\\

%Nucleus               & The nucleus is a membrane enclosed part of the cell. It
%contains the cell's genetic information, in the form of DNA and RNA
%molecules.\\

%\end{tabular}
%\end{flushleft}
%\begin{flushleft}
%\begin{tabular}{l p{0.8\linewidth}}

%rRNA                  & \qt{Ribosomal RNA} describes any of several RNAs which
%become part of the ribosome, and thus are involved in translating mRNA and
%synthesizing proteins.\\

%Shotgun cloning       & The process of randomly shearing an organism's genomic
%DNA and cloning it into a suitable vector, resulting in a genomic library.\\

%Shotgun sequencing    & Sequencing the DNA library created by shotgun cloning.\\

%Signal transduction   & Process in which an extracellular signaling molecule
%activates a specific receptor located in the border (or inside) a cell; in turn
%this receptor triggers a chain of events inside the cell that leads to its
%response.

%Transcription         & The process of copying DNA to produce an RNA transcript.
%This is the first step in the expression of any gene. The resulting RNA will
%produce the desired protein molecule by the process of translation.\\

%Translation           & The process of decoding a strand of mRNA, thereby
%producing a protein based on the code.\\

%tRNA                  & \qt{Transfer RNA} represents one of a class of rather
%small RNAs used by the cell to carry amino acids to the enzyme complex (the
%ribosome) which builds proteins, using an mRNA as a guide.\\

%\end{tabular}
%\end{flushleft}

