\chapter*{Abstract}

The advent of next generation sequencing methods has revolutionized the field of
molecular biology in the past few years. Nowadays, we are able to produce
enormous amounts of biological information, both quickly and at low cost. As
such, tools have to evolve accordingly, in order to cope with such large volumes
of information. In this report we discuss the usage of computer tools capable of
conducting gene expression profiling based on information obtained through
\rnaseq{} techniques, applied to a specific set of biological problems. In
particular, we present the idealization process and implementation details of a
web platform capable of addressing these problems, as well as the actual
platform prototype. This report also includes a literature review, covering both
the biological and technical aspects of the work, with special emphasis in
machine learning techniques applied to data mining tasks. Lastly, we review the
work done and results obtained so far and outline the possible future of the web
platform.

\chapter*{Resumo}

\begin{otherlanguage}{portuguese}
O advento das técnicas de sequenciação de nova geração revolucionou o campo da
biologia molecular nos últimos anos. Hoje em dia somos capazes de produzir
enormes quantidades de informação biológica rapidamente e a baixo custo. Assim
sendo, as ferramentas devem também evoluir, a fim de lidarem com estas extensas
quantidades de informação. Neste relatório discutimos o uso de ferramentas
informáticas capazes de analisar perfis de expressão génica com base em
informação obtida através de técnicas de \textit{\rnaseq{}}, aplicadas a um
conjunto específico de problemas biológicos. Em particular, apresentamos o
processo de idealização e os detalhes de implementação de uma plataforma
\textit{web} capaz de resolver estes problemas, assim como o protótipo funcional
dessa plataforma. Este relatório inclui também uma revisão da literatura,
cobrindo os aspetos biológicos e técnicos deste trabalho, com um ênfase especial
em técnicas de aprendizagem máquina aplicadas a tarefas de \textit{data mining}.
Por fim, revemos todo o trabalho efetuado e os resultados obtidos até ao momento
e delineamos as possibilidades futuras para a plataforma \textit{web}.
\end{otherlanguage}
