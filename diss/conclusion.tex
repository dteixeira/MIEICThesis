\chapter{Conclusions} \label{chap:conclusions}

\section*{}

%\begin{Notes}
%- Check PDIS conclusion.\\
%- Talk about finishing iRAP's web integration and further exploration of its
%results.\\
%- Talk about full automated integration between both tools.\\
%\end{Notes}

In this thesis we have proposed the development of an integrated platform for
RNA-Seq data analysis. This platform is able to perform the entire analysis
process, from the sequencing reads, to grouping genes by their RBPs. This means
that is able to perform read alignment, quantification and differential
expression analysis tasks, as well as data set enrichment, RBP identification
and clustering analysis of genes and proteins. Through two case studies we
showed the developed prototype in use, and assessed its correctness and
efficiency. Below we will share our thoughts on the fulfilment of the objectives
of this thesis, and present some possibilities for any future development in our
solution.

During this thesis we were advised by molecular biology experts from IBMC. The
data set used in the second case study was provided by IBMC experts and the
results obtained are currently being experimentally verified in their
laboratory.

\section{Objective Fulfilment}

Our objectives, in terms of studying the problem at hand and developing a
solution to it, were completely fulfilled. The proposed solution corresponds to
all of our expectations. However, as previously discussed, the implementation of
the RNA-Seq data analysis pipeline system was not completed, due to time
constraints. Integration with PBS Finder's analysis pipeline was not
accomplished. Similarly, it was also not possible to provide the RNA-Seq
analysis pipeline's functionality through the web interface. As such, our
objective of prototyping and testing the complete system could not be completely
achieved.

\section{Future Work}

A natural continuation of the proposed work would be to finish the
implementation and integration of the RNA-Seq data analysis pipeline. This would
allow our solution to work as designed, integrating the complete analysis
pipeline, from sequencing data to gene clustering and result visualization.
Furthermore, it would be useful to study the developed tools in terms of
performance, under large volumes of information and requests. Whilst the tools
were developed taking in consideration their performance, making them available
in a large scale would require another kind of infrastructure.
