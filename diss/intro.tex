\chapter{Introduction} \label{chap:intro}

\section*{}

\begin{Notes}
- Introduction seems fine, check later if emphasis on sequencing should be toned
down.
\end{Notes}

Molecular biology is a branch of biology that studies biological activities of
living beings, at a molecular level. The grounds for this field of study were
set in the early 1930s, although it only emerged in its modern form in the
1960s, with the discovery of the structure of \dna. Among the processes studied
by this branch of biology is gene expression. Gene expression (further explained
in Chapter \ref{chap:sota}) is the process by which \dna{} molecules are
transformed into useful genetic products, typically proteins, which are
essential for living organisms. This knowledge is not only important in fields
like evolutionary or molecular biology, but has crucial applications in fields
such as medicine. One example of such an application is the usage of gene
expression analysis in the diagnosis and treatment of cancer patients
\cite{Pusztai01062003}.

With the advent of \ngs{} (\textit{Next Generation Sequencing}) techniques,
researchers have at their disposal huge amounts of sequencing data, that is not
only cheaper and faster to produce, but also more commonly available. This data
can then be used to obtain relevant information about organisms' gene
expression. But, as the cost of sequencing genomes was reduced, the cost of
processing such information was increased. \ngs{} techniques tend to produce
much smaller reads\footnote{A \textit{read} is a single fragment of a
genome/\trans, obtained through sequencing techniques.} than previously used
techniques, presenting a more complicated problem, from a computational
standpoint \cite{Wolf2013}.

\section{Domain Problem} \label{sec:problem}

\begin{Notes}
- Genome assembly is not part of the work, read alignment is.\\
- Work objectives float around differential expression analysis, gene enrichment
and protein binding site discovery.\\
\end{Notes}

Despite its great advancements in the past decades, molecular biology is still a
relatively new subject and, as such, there are still some unknowns and partial
knowledge in this area. In respect to gene expression, some mechanisms of this
intricate process are yet to be fully understood. One such mechanism is the one
that regulates the transcription speed of \rna. One of the objectives of the
thesis is to understand how the final sequences of a gene's exons are
responsible for the speed at which the exons themselves are transcribed. The
other objective is to understand how RNA binding protein (RBP) manipulation can
be used to better understand an organism's gene expression. This are, however,
complex tasks that can be further decomposed in the three main problems that
will be addressed in the thesis, namely:

\begin{itemize}

  \item
  Sequencing read alignment against a reference genome and differential
  expression analysis between samples of different individuals (of the same
  species). This is effectively one of the most complex problems addressed in
  the thesis. We will use data obtained through a sequencing method called
  \rnaseq{}\footnote{\rnaseq{} is also referred to as \textit{Whole \Trans{}
  Shotgun Sequencing}, or WTSS.}. Further insight about this method will be
  given in Chapter \ref{chap:sota}, with particular emphasis for tools used to
  align and analyze this data (Section \ref{sec:seqtools}).

  \item
  Gene enrichment and RBP analysis. This part of the work aims to collect as
  much relevant information as possible about the particular genes being studied
  at the time, to help biologists better understand their function. RBP
  knowledge is particularly important for gene manipulation and a great tool for
  better understanding gene expression, as will be further described in Chapter
  \ref{chap:sota}.

  \item
  Further analysis of the produced data, using machine learning techniques
  applied to data mining, specifically to clustering analysis. These techniques
  will be employed in an effort to try to give biologists more relevant
  information about gene expression, uncovering possible relationships in the
  retrieved information. This topic will be developed in Section
  \ref{sec:mlearning}.

\end{itemize}

Solving these problems requires the use of computational tools. As such, the
development of a computer system (or multiple systems) to tackle these problems
emerges as a secondary objective of the thesis. The details of the idealization
of this system will be presented in Chapter \ref{chap:description}, while its
concrete implementation will be discussed in Chapter \ref{chap:implementation}.

\section{Motivation and Objectives} \label{sec:motivation}

\begin{Notes}
- Check previous notes, genome assembly is not needed.\\
- Don't forget to mention the gene enrichment/protein binding site analysis.
- Talk about how simplicity is needed because many times biologists don't
understand the tools already available.
\end{Notes}

As mentioned above, the assembly of a \trans{} is a very complex problem, even
more so when \rnaseq{} is used. Aligning such large amounts of data in the form
of small reads to a complete (and therefore extensive) genome is a complicated
task. It is, however, an essential problem that needs to be solved, as without
an assembled \trans{} there is no analyzable data. In turn, this analysis of the
\trans{} is essential to further scientific development in fields like
molecular biology and medicine, as stated above. With this project we aim at
developing an automated computer system, capable of solving these problems. More
than applying exclusively to the particular problem at hand, we want to develop
an useful and intuitive solution, that might easily be used by researchers in
this field of study to solve similar problems, extending its benefits to a
broader scientific community.

\section{Project} \label{sec:project}

\begin{Notes}
- Change project description to fit the alignment/differential expression
pipeline and the PBS tool.\\
- Refer both tools as independent, but mention that they're supposed to be
integrated with each other.\\
\end{Notes}

The project itself will revolve around the development of a prototype computer
system. The first objective of this prototype is to solve the aforementioned
problems, namely the \trans 's assembly and analysis. Beyond this objective, the
prototype should become an useful and intuitive tool for any researcher
investigating this or similar problems. To fulfill these objectives, we will
need to develop a complex system, composed by several smaller systems.
Therefore, the envisioned system architecture is divided into three major
components, to wit:

\begin{description}

  \item[Information system]
  is responsible for storing and managing genetic data, coordinating interaction
  between the other components of the system and providing a web interface for
  user interaction. This component will be based mainly on typical web
  technologies, that is, relational databases for data storage (SQL DBMSs), web
  frameworks for business logic implementation (Ruby on Rails, Padrino,
  NodeJS\footnote{Ruby on Rails and Padrino are Ruby based web frameworks, while
  NodeJS is a Javascript based web framework.}) and web markup and styling
  languages for interface implementation (HTML, CSS).

  \item[Assembly pipeline]
  will use genetic data stored in the information system in order to produce
  assembled \trans s. This pipeline will be composed by several tools,
  corresponding to each phase of the assembly process, possibly intercalated
  with data format conversion programs. The tools to be used in this component
  will be further discussed in Section \ref{sec:seqtools}.

  \item[\Trans{} analysis]
  will be responsible for the data mining analysis of the assembled \trans s, in
  the context of the problem of the thesis. It is expected that this component
  integrates with the rest of the system. Further information about the tools
  that will be used in this component is given in Section \ref{sec:mintools}.

\end{description}

From here, this document will not dwell in the details of the implementation of
such a system, focusing instead the specificities of the problem's solution,
from the molecular biology and data mining perspectives. This is due to the fact
that the development of the system itself is not the focus of the thesis, but
rather a natural consequence of the project's work process.

\section{Structure of the Report} \label{sec:outline}

\begin{Notes}
- Update this last.\\
\end{Notes}

Besides the introduction chapter, this document is composed by three additional
chapters. Chapter \ref{chap:sota} introduces some basic biology and RNA-Seq
concepts, that are essential to understand the problems with which this document
deals. Furthermore, we describe the main techniques used for genome/\trans{}
sequencing and assembly, their differences, applications and the tools and data
formats typically used in those areas. Lastly, we give some insight about data
mining algorithms and how they will be applied in the context of the project.
Chapter \ref{chap:workplan} outlines the main steps in the development of the
project (and the respective software prototype) and attempts to provide a
feasible schedule for the work's execution. It also presents the datasets that
will be studied and used in this work, their origins and features, as well as
the validation methods that will be used to ascertain the quality of our
results. Chapter \ref{chap:conclusions} sums up the what has been defined in the
report, emphasizing the problem that the thesis addresses and the work that will
be executed towards solving that problem. It will also give a brief idea of what
are the expected results at the end of the project.
