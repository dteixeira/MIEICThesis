\chapter{Conclusions}\label{chap:conclusions}

In this chapter we will review the general objectives of the thesis. We will
also recap the idea behind the proposed project, along with its implementation
idealization. Lastly, we will give some perspective about the future work.

\section{Objectives}

As stated before, our main objective for this thesis is to study the impact of
an exon's final sequence in its own transcription speed. This objective can be
decomposed in three base objectives. These objectives are genetic data
collection, \trans{} assembly and \trans{} analysis.

\section{Project}

The project will be the materialization of the aforementioned objectives. As
such, it will be composed by three main components. The first component answers
the genetic data collection problem, implementing an information system and
respective web interface, for data storage and manipulation. The second
component will be an assembly pipeline, that will assemble \trans s using the
information system. Finally, we will have a data analysis module, that will
apply data mining techniques to analyse the assembled \trans s.

\section{Future Work}

The project will be implemented in two main phases, first implementing the base
information system and the assembly pipeline and then using them, together with
chosen data mining techniques, to produce and analyse \trans s. These results
will be evaluated using both our defined evaluation methods and expert opinions
from IBMC, being the latter the final measure of the project's success.
