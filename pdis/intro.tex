\chapter{Introduction} \label{chap:intro}

\section*{}

\section{Context and Motivation} \label{sec:context}

\section{Objectives} \label{sec:goals}

\section{Document Outline} \label{sec:outline}

Besides the introduction chapter, this document is composed three additional
chapters. These chapters have the following structure:

\begin{description}

  \item[Chapter \ref{chap:sota}]
  introduces some basic Biology and \rnaseq{} concepts, that are essential to
  understand the problems with which this document deals. Furthermore, we
  describe the main techniques used for genome/transcriptome sequencing and
  assembly, their differences and applications and the tools and data formats
  typically used on those areas. Lastly, we give some insight about machine
  learning algorithms and how they will be applied to this work.

  \item[Chapter \ref{chap:validation}]
  presents the datasets that will be studied and used in this work, their
  origins and features. We will also refer the validation methods that will
  be used to access the quality of our results.

  \item[Chapter \ref{chap:workplan}]
  outlines the main steps in the development of this thesis (and the respective
  software prototype). In the last part of this chapter we will attempt to
  provide a feasible schedule for this work's execution.

\end{description}
