\chapter{Introduction} \label{chap:intro}

\section*{}

This chapter aims at giving a general overview about the themes address by this
thesis. We will address the context in which the thesis is inserted, as well as
the motivation that led to its proposal. Furthermore there will be brief
description of this thesis' main objectives and the methods that will be used to
achieve those objectives.

\section{Context and Motivation} \label{sec:context}

- explain gene expression

- explain importance and applications of gene expression profilling

- explain that nowadays sequencing data is easier and cheaper to obtain, but
  harder to process

- explain that there are several techniques to obtain gene expression
  information

- explain that in the thesis only RNA-Seq will be analysed

\section{Objectives} \label{sec:goals}

While defining the concrete objectives of this thesis it becomes relevant to
separate them in two groups: strictly biology research related objectives and
more general, software solution development objectives. Despite this division,
both objectives are tightly interconnected, and each complements the other.

From a molecular biology standpoint, the main objective of this thesis will be
to try to understand the mechanisms that regulate the speed of transcription for
coding regions of the \dna. This information will be obtained using the
\rnaseq{} method, that will be further discussed in Chapter \ref{chap:sota}.
There are several intermediate objectives for this particular problems, as
follows:

\begin{itemize}

  \item
  Alignment of the given sequencing reads into a known reference genome. This is
  one of the first steps in the \rnaseq{} process and is effectively one of the
  most complex problems addressed by this thesis. Some of the tools used in this
  particular step of the process will be referenced in Section \ref{sec:seqtools}.

  \item
  Further analysis of the \rnaseq{} results using machine learning algorithms,
  applied to data mining. These techniques will be used in an effort to try to
  understand the already mentioned transcription mechanisms. This topic will be
  developed in Section \ref{sec:mlearning}.

\end{itemize}

The last objective of this thesis is the development of a software platform
prototype. This prototype comes as a materialization of the work done along the
previous objectives, combining the developed genetic data processing pipeline,
with a web information system and with data mining tools. When completed, the
prototype should allow for users to store, search and manipulate their genome
sequencing data. This data can them be assembled using the tool pipeline
developed for the analysis of our own experimental dataset. Lastly, the
prototype should integrate data mining tools, that would allow users to
reproduce the types of data analysis that were done in this thesis, on their own
results.

This document, however, will not dwell in the details of the implementation of
such a platform, but rather in the molecular biology section of the overall
problem. This is largely due to the fact that the development of the web
platform is highly dependent on the tools and methods that will be used for
tackling the biology aspects of the problem and, as such, is likely to suffer
significant alterations.

\section{Document Outline} \label{sec:outline}

Besides the introduction chapter, this document is composed three additional
chapters. These chapters have the following structure:

\begin{description}

  \item[Chapter \ref{chap:sota}]
  introduces some basic Biology and \rnaseq{} concepts, that are essential to
  understand the problems with which this document deals. Furthermore, we
  describe the main techniques used for genome/transcriptome sequencing and
  assembly, their differences and applications and the tools and data formats
  typically used on those areas. Lastly, we give some insight about machine
  learning algorithms and how they will be applied to this work.

  \item[Chapter \ref{chap:validation}]
  presents the datasets that will be studied and used in this work, their
  origins and features. We will also refer the validation methods that will
  be used to access the quality of our results.

  \item[Chapter \ref{chap:workplan}]
  outlines the main steps in the development of this thesis (and the respective
  software prototype). In the last part of this chapter we will attempt to
  provide a feasible schedule for this work's execution.

\end{description}
