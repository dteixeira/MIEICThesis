\chapter{Introduction} \label{chap:intro}

% TODO REMOVE
New structure idea:\\
  - Context: same as context and motivation right now;\\
  - Motivation and objectives: talk about exon analysis, and other high level
    objectives for the project; just reference the technical aspects of the
    problem, a let the bulk prototype be described in project;\\
  - Project: describe the project itself, the prototype and development phases;
    refer to planning;\\
  - Structure: it's already good;\\

\section*{}

This chapter aims at giving a general overview about the themes address by this
thesis. We will address the context in which the thesis is inserted, as well as
the motivation that led to its proposal. Furthermore there will be brief
description of this thesis' main objectives and the methods that will be used to
achieve those objectives.

\section{Context and Motivation} \label{sec:context}

Molecular biology is a branch of biology that studies biological activities of
living being, at a molecular level. The early grounds for this field of study
were set in the early 1930's, although only emerging in its modern form in the
1960's, with the discovery of the structure of \dna. Among the processes studied
by this branch of biology is gene expression. Gene expression is the process by
each \dna{} molecules are transformed into useful genetic products, typically
proteins, which are essential for living organisms. This knowledge is not only
important in fields like evolutionary biology or molecular biology, but may have
crucial applications in fields such as medicine. One example of such an
application is the usage of gene expression analysis in the treatment of cancer
patients \cite{Pusztai01062003}.

With the advent of \ngs{} (Next Generation Sequencing) techniques, researchers
have at their disposal huge amounts of sequencing data, that is not only cheaper
and faster to produce, but also more commonly available. This data can then be
used to obtain relevant information about organisms' gene expression. But, as
the cost of sequencing genomes was reduced, the cost of processing such
information was increased. \ngs{} techniques tend to produce much smaller
reads\footnote{A \textit{read} is a single fragment of a genome/transcriptome,
obtained through sequencing techniques.} than previously used techniques, which
present a much harder problem, from a computational standpoint \cite{Wolf2013}.

\section{Objectives} \label{sec:goals}

While defining the concrete objectives of this thesis it becomes relevant to
separate them in two groups: strictly biology research related objectives and
more general, software solution development objectives. Despite this division,
both objectives are tightly interconnected, and each complements the other.

From a molecular biology standpoint, the main objective of this thesis will be
to try to understand the mechanisms that regulate the speed of transcription for
coding regions of the \dna, in other words, to understand the mechanisms that
regulate gene expression. This information will be obtained using the \rnaseq{}
method, that will be further discussed in Chapter \ref{chap:sota}. There are
several intermediate objectives for this particular problems, as follows:

\begin{itemize}

  \item
  Alignment of the given sequencing reads into a known reference genome. This is
  one of the first steps in the \rnaseq{} process and is effectively one of the
  most complex problems addressed by this thesis. Some of the tools used in this
  particular step of the process will be referenced in Section \ref{sec:seqtools}.

  \item
  Further analysis of the \rnaseq{} results using machine learning algorithms,
  applied to data mining. These techniques will be used in an effort to try to
  understand the already mentioned transcription mechanisms. This topic will be
  developed in Section \ref{sec:mlearning}.

\end{itemize}

The last objective of this thesis is the development of a software platform
prototype. This prototype comes as a materialization of the work done along the
previous objectives, combining the developed genetic data processing pipeline,
with a web information system and with data mining tools. When completed, the
prototype should allow for users to store, search and manipulate their genome
sequencing data. This data can them be assembled using the tool pipeline
developed for the analysis of our own experimental dataset. Lastly, the
prototype should integrate data mining tools, that would allow users to
reproduce the types of data analysis that were done in this thesis, on their own
results.

This document, however, will not dwell in the details of the implementation of
such a platform, but rather in the molecular biology section of the overall
problem. This is largely due to the fact that the development of the web
platform is highly dependent on the tools and methods that will be used for
tackling the biology aspects of the problem and, as such, is likely to suffer
significant alterations.

\section{Structure of the Report} \label{sec:outline}

Besides the introduction chapter, this document is composed three additional
chapters. These chapters have the following structure:

\begin{description}

  \item[Chapter \ref{chap:sota}]
  introduces some basic Biology and \rnaseq{} concepts, that are essential to
  understand the problems with which this document deals. Furthermore, we
  describe the main techniques used for genome/transcriptome sequencing and
  assembly, their differences and applications and the tools and data formats
  typically used on those areas. Lastly, we give some insight about data mining
  algorithms and how they will be applied to this work.

  \item[Chapter \ref{chap:workplan}]
  outlines the main steps in the development of this thesis (and the respective
  software prototype) and attempts to provide a feasible schedule for the work's
  execution. It also presents the datasets that will be studied and used in this
  work, their origins and features, as well as the validation methods that will
  be used to ascertain the quality of our results.

  \item[Chapter \ref{chap:conclusions}]
  sums up the what has been defined in the report, emphasizing the problem that
  the thesis addresses and the work that will be executed towards solving that
  problem. It will also give a brief idea of what are the expected results at
  the end of the project.

\end{description}
